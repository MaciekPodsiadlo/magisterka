\chapter{Przetwarzanie obrazu}
\label{cha:przetwarzanieObrazu}

\section{Parametry kamery}
\label{sec:parametryKamery}

\section{Redukcja dystorsji}
\label{sec:redukcjaDystorsji}

\section{Korekcja Cienia}
\label{sec:korekcjaCienia}

Tak zwana korekcja cienia była operacją arytmeczną na obrazie. Miała ona za zadanie usunięcie regularnych odchyleń od jasności. Mogły być one wynikiem między innymi przez: efekt cienia, nieprostopadłe oświetlenie czy zaciemnianie brzegów spwodowane mniejszą przepuszczalnością światła na krańcach soczewki. W wykonanych eksperymentach nie zaobserwowano ostatniego przypadku. Korekcję osiągnięto przez zastosowanie różnicy między badanym obrazem, a obrazem tła w tych samych warunkach oświetleniowych. Wartość każdego piksela została obliczona na podstawie wzoru:
$b(x,y) = (255 + ( g(x,y) - g_r(x, y) ) ) / 2$
Gdzie: 
b(x,y) - jasność piksela w obrazie wynikowym;
g(x,y) - jasność piksela w obrazie badanym;
$g_r$(x,y) - jasność piksela w obrazie tła;

\section{Filtracja}
\label{sec:filtracja}

\section{Wyznaczenie konturów przedmiotów}
\label{sec:wyznaczenieKonturow}


